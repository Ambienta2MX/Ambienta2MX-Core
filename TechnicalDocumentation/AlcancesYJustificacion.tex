\chapter {Definici\'on tem\'atica de Ambienta2MX}
  \section {Objetivo general}
    \paragraph {Considerando el problema de informaci\'on y estructura de datos que tiene M\'exico en la actualidad, adem\'as de seguir la tendencia y aprovechar la brecha que ha disminuido el uso de datos abiertos, el equipo de \underline{Ambienta2MX} decidi\'o afrontar la tarea de desarrollar una herramienta que permita la conceptualizaci\'on de la informaci\'on de variables ambientales e indices de calidad y contaminaci\'on del aire que el INEGI y otras instituciones p\'ublicas o privadas almacenan y/o exponen para fines educativos, informativos o de uso part\'icular.}
  
  \section{Justificación}
    \paragraph{Existe la necesidad de una fuente de información que presente datos climáticos y de contaminación del aire que ofrecen diversas instituciones como el Instituto Nacional de Estadística y Geografía (INEGI), el Servicio Meteorólogico Nacional y la Comisión Nacional del Agua de manera estándarizada.
               Esto con la finalidad de ordenar y categorizar un conjunto de metadatos.}
    \paragraph{La aplicación del esqueleto de Ambienta2MX, puede ser tomada como base para la expansión hacia otro tipo de variables y servicios, quedando a disposición de los futuros interesados analizar los datos que se almacenarán en las bases de datos de tipo MX (mencionadas más adelante) además de llevar a cabo el proceso de búsqueda y obtención de los datos deseados por éstos.}
    \subsection{Alcances en Trabajo Terminal 2}
    \paragraph{Considerando la metodología de trabajo, además de la ambición del proyecto, se decidió delimitar la información climatológica y de variables de contaminación al Distrito Federal cómo caso de estudio, principalmente contando con datos de la zona norte de la capital del país.}
    \paragraph{Como muestra visual del caso de estudio, se proveerán mapas que demostrarán la aplicación de los datos estandarizados y obtenidos de diversas fuentes, cabe destacar, que la información puede tener diversas interpretaciones dependiendo el uso que se le vaya a dar, queda a disposición del usuario final la interacción y aplicación del contenido climatológico recopilado por la plataforma Ambienta2MX.}

