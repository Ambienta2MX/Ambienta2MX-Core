\chapter {Definici\'on tem\'atica de Ambienta2MX}
  \section {Objetivo general}
    \paragraph {Considerando el problema de informaci\'on y estructura de datos que tiene M\'exico en la actualidad, adem\'as de seguir la tendencia y aprovechar la brecha que ha disminuido el uso de datos abiertos, el equipo de \underline{Ambienta2MX} decidi\'o afrontar la tarea de desarrollar una herramienta que permita la conceptualizaci\'on de la informaci\'on de variables ambientales e indices de calidad y contaminaci\'on del aire que el INEGI y otras instituciones p\'ublicas o privadas almacenan y/o exponen para fines educativos, informativos o de uso part\'icular.}
  
  \section{Justificación}
    \paragraph{Existe la necesidad de una fuente de información que presente datos climáticos y de contaminación del aire que ofrecen diversas instituciones como el Instituto Nacional de Estadística y Geografía (INEGI), el Servicio Meteorólogico Nacional y la Comisión Nacional del Agua de manera estándarizada.
               Esto con la finalidad de ordenar y categorizar un conjunto de metadatos.}
