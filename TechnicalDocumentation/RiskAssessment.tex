\chapter{Análisis y gestión de riesgos}
  \section{Definición y clasificación}
    \paragraph{Los objetivos de la gestión de riesgos son identificar, controlar y eliminar las fuentes de riesgo antes de que empiecen a afectar el cumplimiento de los objetivos del proyecto.}
    \paragraph{El riesgo siempre implica una incertidumbre y una pérdida potencial, es necesario llevar una cuantificación de estos parametros, por lo que suelen ser clasificados en diferentes categorias por ejemplo:}
    \begin{itemize}
      \item Riesgos del proyecto
      \item Riesgos técnicos
      \item Riesgos de negocio
    \end{itemize}
    \paragraph{Es necesario llevar un proceso de administración de riesgos y planes de contingencia para poder controlar esos inesperados eventos. Primero se tienen que identificar los riesgos, considerando principalmente los más potenciales, posteriormente analizarlos y dandoles una prioridad, generar planes de contingencia para finalmente supervizarlos y actuar conforme a lo acordado.}
    \paragraph{Los riesgos pueden ser clasificados respecto a estudios previos o con relación a la experiencia de trabajo que tengan los desarrolladores y analistas del proyecto, sin embargo, generalmente suelen asignarseles ciertas ponderaciones por convención, como se muestra a continuación:}
    \begin{itemize}
      \item Muy bajo ( < 10\% )
      \item Bajo ( 10 - 25\% )
      \item Moderado (25 - 50\% )
      \item Alto (50 - 75\% )
      \item Muy Alto ( > 75\% )
    \end{itemize}
    \paragraph{Todos los riesgos deben encontrarse también clasificados entre cualquiera de las cuatro valoraciones: Insignificante, Tolerable, Serio, Catastrófico; los planes de contingencia suelens ser desarrollados para aquellos riegos con probabilidad de moderada a muy alta considerando, tomando en cuenta un impacto serio o castastrófico.}
    \paragraph{A lo largo del desarrollo de Ambienta2MX surgieron y surgirán eventos y sucesos inesperados que serán considerados como riesgos potenciales dependiendo el impacto que estos lleguen a tener.}
    \begin{table}[h]
    \centering
      \begin{tabular}{lllll}
        \hline
        \multicolumn{5}{c}{{\bf Tabla de Riesgos}} \\ \hline
        \multicolumn{1}{|l|}{{\bf Descripcion}} & \multicolumn{1}{l|}{{\bf Tipo de Riesgo}} & \multicolumn{1}{l|}{{\bf Valoración}} & \multicolumn{1}{l|}{{\bf Probabilidad}} & \multicolumn{1}{l|}{{\bf Plan de acción}} \\ \hline
        \multicolumn{1}{|l|}{Riesgo1} & \multicolumn{1}{l|}{Proyecto} & \multicolumn{1}{l|}{Tolerante} & \multicolumn{1}{l|}{10\%} & \multicolumn{1}{l|}{Ninguno} \\ \hline
        \multicolumn{1}{|l|}{Riesgo2} & \multicolumn{1}{l|}{Técnico} & \multicolumn{1}{l|}{Tolerante} & \multicolumn{1}{l|}{10\%} & \multicolumn{1}{l|}{Ninguno} \\ \hline
        \multicolumn{1}{|l|}{Riesgo3} & \multicolumn{1}{l|}{Negocio} & \multicolumn{1}{l|}{Serio} & \multicolumn{1}{l|}{20\%} & \multicolumn{1}{l|}{Actuar} \\ \hline
        \multicolumn{1}{|l|}{Riesgo4} & \multicolumn{1}{l|}{Proyecto} & \multicolumn{1}{l|}{Serio} & \multicolumn{1}{l|}{10\%} & \multicolumn{1}{l|}{Actuar} \\ \hline
        \multicolumn{1}{|l|}{Riesgo5} & \multicolumn{1}{l|}{Negocio} & \multicolumn{1}{l|}{Catastrófico} & \multicolumn{1}{l|}{10\%} & \multicolumn{1}{l|}{Actuar} \\ \hline
      \end{tabular}
      \caption{Analisis de Riesgos}
      \label{Analisis de riesgos}
    \end{table}