\chapter {Descripción y Módulos de Ambienta2MX}
  \section {¿Qué y para qué es Ambienta2MX?}
    \paragraph {\underline{Ambienta2MX} es el nombre de la plataforma que formará parte de una macro solución orientada a la estandarización de metadatos que el INEGI y otras instituciones públicas.}
    \paragraph{Actualmente no existe un estándar de datos geográficos a nivel nacional. Han existido aproximaciones mediante concursos que instituciones públicas como el INEGI ha publicado, o simplemente han existido propuestas que han brindado una solución incompleta a la unión y manejo de información geográfica, geodésica, hidrográfica, climática, topográfica, etc.}
    \paragraph{\underline{Ambienta2MX} toma parte de todo el problema y propone una infraestructura lógica para afrontar la estandarización de variables ambientales y algunos índices de contaminación. Esta información actualmente se encuentra en formatos muy rudimentarios como textos planos sin algún protocolo o formato de interpretación.}
    \paragraph{Sistemas semejantes, por ejemplo, el \textbf{Servicio Meteorológico Nacional} carece de algún recurso del cual se puedan realizar consultas que no sea mediante su portal web, esto trae problemas directos de compatibilidad con otros sitemas. Un caso semejante tenemos con la información que la \textbf{Conagua} maneja en sus centrales meteorológicas a lo largo del país, los datos que brindan se actualizan de forma periodica y el único medio de acceso es a través de una página de internet que devuelve archivos en formato de texto u hojas de cálculo.}
    \paragraph{Los impedimentos antes mencionados conllevan a situaciones tan triviales como la consulta de datos para algúna región o punto específico del territorio nacional, al existir diversas fuentes no es posible tener un compendio del cual tomar la información que más nos convenga. Si a este problema se le añade que los datos carecen de un estandar, llegamos al punto en el que intentar manipular o tratar los datos se vuelve una tarea complicada y en exceso tediosa.}
    \paragraph{Considerando dichos problemas \underline{Ambienta2MX}, propone un estandar de datos climáticos tomando como referencia diversas fuentes y adaptando los tipos de datos a tecnologías y tendencias actuales, brindando así una mayor portabilidad y simplicidad en la consulta de información.}
  \section{Diagrama de Ambienta2MX}
  
    \paragraph{Párrafo chistoso}
  \section{Otra cosa más}
    \paragraph{Existe la necesidad de una fuente de datos que presente información climática}