%
% External sources
%

\newpage
\section*{Fuentes externas de datos}
\paragraph{Una de las fuentes usadas fue la de \textbf{CONAGUA}, institución de caracter público y mantenida por el Gobierno Federal, cuya misión es preservar las aguas nacionales y bienes públicos para su administración sustentable.}
\paragraph{Una de las funciones de la CONAGUA es contar con los registros de clima e información ambiental de las fuentes que tiene distribuidas a lo largo del territorio nacional, éstas fuentes carecen de un formato definido lo cual complica en gran forma el análisis de datos.\cite{32}}
\paragraph{Para el desarrollo de la etapa final del proyecto se procedió a realizar un scrapping de la información expuesta por sus servicios y adaptarlos al modelo de datos propuesto por el equipo de trabajo.}
\paragraph{A continuación se muestra un ejemplo de la información brindada por los servicios de la CONAGUA.}
\lstinputlisting[language={}]{../Resources/Conagua.txt} 
\addcontentsline{toc}{chapter}{Anexo 3: Fuentes de datos externas}