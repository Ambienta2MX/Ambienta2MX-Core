% Nginx Instalation process
%
% External sources
%

\newpage
\section*{Instalación de Nginx}
	\paragraph{Para sistemas basados en Unix (Linux y Mac).}
	\paragraph{La instalación se puede realizar desde el gestor de paquetes dependiendo la distribución a usar, el ejemplo actual se encuentra en un sistema de tipo Ubuntu en su versión 14.04.}
	\paragraph{Sólo basta con instalar Nginx desde la línea de comandos utilzando el siguiente comando}
	\begin{alltt}
		sudo apt-get install nginx -y
	\end{alltt}
	\paragraph{Esto bajará las bibliotecas, archivos fuente y dependencias necesarias para instalar Nginx, posteriormente se procede a la edición del archivo de configuración para generar la unión con los demás módulos de Ambienta2MX.}
	\paragraph{Se tendrá que modificar el archivo ubicado en ``/etc/nginx/sites-enabled/default'', agregando líneas o bien reemplazando el contenido del archivo con lo que se muestra a continuación:}
	\begin{alltt}
		upstream api_fasteagle \{
		 server    127.0.0.1:7777;
		\}
		server \{
		 listen 80 default_server;
		 listen [::]:80 default_server ipv6only=on;
		 root /usr/share/nginx/html;
		 index index.html index.htm;
		 server_name localhost;

		 location / \{  
		  try_files $uri $uri/ =404;
		 \}

		 location /api/fasteagle \{
		    proxy_set_header X-Real-IP $remote_addr;
		    proxy_set_header X-Forwarded-For $proxy_add_x_forwarded_for;
		    proxy_set_header Host $http_host;
		    proxy_set_header X-NginX-Proxy true;
		    rewrite ^/api/fasteagle/?(.*) /$1 break;
		    proxy_pass http://api_fasteagle;
		    proxy_redirect off;
		  \}  
		\} 
	\end{alltt}
	\paragraph{La configuración de Nginx puede ser replicada para cualquier servicio o servidor externo, utilizando la directiva upstream, que funciona como un proxy entre el servidor web y los servidores de aplicación.}
\addcontentsline{toc}{chapter}{Anexo 6: Instalación y configuración de Nginx} 