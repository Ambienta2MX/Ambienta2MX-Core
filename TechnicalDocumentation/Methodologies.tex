\clearpage
\chapter{Metodologías}

\section{Desarrollo ágil de software}
  \paragraph{El desarrollo ágil propone una alternativa al desarrollo de software tradicional. Los enfoques del desarrollo ágil son principalmente usados en el desarrollo de software para ayudar a las compañias a responder fácilmente al cambio.}

\subsection{Scrum}
  \paragraph{Scrum es un marco de gestión para el desarrollo incremental de un producto que proporciona una estructura de roles, reuniones, reglas y artefactos.}
  \paragraph{Scrum utiliza iteraciones de longitud fija denominados Sprints, que son típicamente de dos semanas o 30 días de duración. Los equipos de Scrum intentan generar un incremento de producto potencialmente entregable (debidamente probado) en cada iteración.}
  
 \subsection{Programación Extrema (XP)}
  \paragraph{Es un enfoque disciplinado para entregar software de alta calidad rápida y continuamente. Promueve una alta participación del cliente, retroalimentación rápida, pruebas continuas, planeación continua y entrega de software funcional en intervalos muy frecuentes que van de 1 a 3 semanas.}
  
\section{Técnicas de desarrollo ágil}

\subsection{Desarrollo guiado por pruebas (Test Driven Development)}


\section{Justificación}

A lo largo del desarrollo del proyecto se utilizarán prácticas características de varios frameworks y metodologías para el desarrollo ágil.

Principalmente plantearemos las historias de usuario que describan los flujos principales de la aplicación y posteriormente desarrollaremos entregables funcionales en plazos de tiempo conocidos como sprints en donde asignaremos un conjunto de actividades a cada integrante del equipo. Se realizará cada módulo utilizando siguiendo el desarrollo guiado por pruebas para asegurar la calidad del software. 


