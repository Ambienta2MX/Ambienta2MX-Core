\subsection{Friendly Dolphin}
    \subsubsection{Definición}
      \paragraph{Éste módulo es el encargado de brindar la información procesada al usuario a través de una página de internet. Es el módo visual que los usuarios finales tendrán para poder interactuar con el ecosistema Ambienta2MX.}
      \paragraph{Se presenta cómo un módulo web que consumirá la información procesada y almacenada por las cuatro bases (MX1,MX2,MX3,MX4) y la base de soporte (Places).}
      \paragraph{La principal función es la de consulta y visualización de datos. Es la capa más expuesta y menos técnica de Ambienta2MX ya que es la que tendrá interacción directa con usuarios no técnicos, sin embargo, contará con los procesos necesarios para poder extraer información de las demás plataformas en formatos convencionales cómo JSON o CSV para uso posterior del usuario.}
      \paragraph{Interactua de forma directa con el bloque \textbf{\emph{Cute Bunny}}, que forma parte de la segunda capa de exposición de datos de Ambienta2MX. Se comunica con los demás módulos mediante servicios de tipo REST que funcionan bajo el patrón de convención sobre configuración\cite{8}, brindando así una gran compatibilidad con éstos además de disminuir el tiempo de desarrollo debido a que no es necesario generar código único y se opta por la reutilización de éste además de apoyarse con el uso de bibliotecas que siguen el mismo método de trabajo.}
      \paragraph{Friendly Dolpin contará con varios procesos y módulos a ser desarrollados. Éste módulo se desarrollará usando tecnologías cómo HTML, Javascript y CSS, además de contar con un ciclo continuo de desarrollo usando herramientas de apoyo cómo Yeoman, Gulp para el maquetado y gestión de tareas comunes en projectos de tipo web.}
      \paragraph{Se hará uso del servidor interno que ofrece Gulp junto con las tareas y gestión de bibliotecas de terceros. En cuanto al desarrollo de los estilos necesarios para las vistas se implementará Bootstrap cómo maquetado CSS y finalmente el manejo de vistas, peticiciones y lógica dentro del navegador de los clientes se implementará un patrón de tipo SPA (Single Page Application)\cite{36} desarrollado por el equipo de trabajo.}
  \newpage
    \begin{landscape}
    \subsubsection{Diagrama por bloques}
   \paragraph{A continuación se muestra el diagrama básico de la aplicación:}
      \begin{figure}[b!]
      \centering
      \includegraphics[width=22.5cm,height=12cm]{./images/DiagramaFriendlyDolphin.png}
      \caption{Diagrama General de Friendly Dolphin}
    \end{figure}
    \end{landscape}
  \newpage