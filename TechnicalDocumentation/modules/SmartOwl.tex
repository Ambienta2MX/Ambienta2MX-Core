\subsection{Smart Owl}
  \subsubsection{Definición}
  \paragraph{La función principal del Smart Owl es obtener información de diversas fuentes que proveen datos climáticos y de contaminación y exponer esta información de manera estandarizada con una estructura definida en formato JSON.}
  \paragraph{Todas la información que se encontrará en las bases de datos de tipo MX será obtenida a través de Smart Owl, muchas de las fuentes no cuentan con los datos climatológicos completos, principalmente las gubernamentales como: CONAGUA, INEGI o SMN, por citar algunas.}
  \paragraph{Considerando esa problematica, Smart Owl busca y trata de resolver la información de los campos faltantes tomando como base distintas fuentes de datos, algunas establecidas y otras de tipo gubernamental.}
  \paragraph{También se tomará información de fuentes que tipo dinámica,es decir, cuya información suele ser actualizada en entre periodos de una o dos horas. Estas fuentes suelen contar con RESTFul API's para consumo de forma programática.}
  \paragraph{El modelo de datos final podrá ser entonces persistido después de que haya sido resuelto completa o parcialmente la petición.} 
  \paragraph{Se llevará el control de los metadatos considerando su origen, su fecha y algunos tags relacionados los que proveen la información.}

  \newpage   
  \subsection{Modelo de datos} 
  \paragraph{El siguiente diagrama de clases muestra el modelo que define el estándar de los datos.}
    \begin{figure}[b!]
    \begin{center}
      \includegraphics[width=14cm,height=10cm]{./images/SmartOwl_ClassDiagram}
      \caption{Diagrama de clases}
    \end{center}
    \end{figure}
  \newpage
  \subsection{Fuentes para la contrucción de la API}
    \paragraph{Hasta ahora se cuenta con 3 recursos principales para obtener los datos del modelo: los archivos que provee la CONAGUA (Comisión Nacional del Agua), la información de Weather Underground y la API de Forecast.io.}
     
  \subsection{Proceso de desarrollo} 
    \paragraph{Una vez que se definieron las fuentes principales para la obtención de datos se escribió una prueba que verifica la obtención de la información de cada una de ellas.} 
    \paragraph{La primera fuente de la que se extrajo información fueron los archivos que publica cada 10 minutos la página de la CONAGUA en el sitio \textbf{http://smn.cna.gob.mx/emas/}.}
    \paragraph{La técnica de Web Scraping fue utilizada para la extracción de los archivos que generan las diferentes estaciones ubicadas en los estados de las república.}
    \paragraph{Para ello se busco la url de cada archivo generado en todas las estaciones del país con ayuda de la biblioteca \textbf{tagsoup}.}
    \paragraph{Después de obtener las urls se persistieron en una base de datos no relacional de tipo llave-valor, en donde la llave es el valor de la latitud y longitud de la estación que genera la información}
    \begin{figure}[b!]
    \begin{center}
      \includegraphics[width=14cm,height=10cm]{./images/DF_Stations}
      \caption{Estaciones con información climática del Distrito Federal}
    \end{center}
    \end{figure}
    \paragraph{Este proceso se ejecuta una sola vez en la aplicación, ya que cuando se despliega se verifica que ya existan las urls de los archivos en la base de datos}
    \paragraph{Ya que se tienen la información de los archivos disponible, se toma busca en la base la url del que tenga la latitud y longitud más cercana a los parámetros de consulta de la API para descargarlo y posteriormente iniciar el proceso extracción de las variables} 
    \paragraph{La segunda fuente con la que se intenta complementar el modelo es Weather Underground.}
    \paragraph{Este sitio expone un servicio que recibe un código de ciudad para obtener la información climática.}
    \begin{figure}[b!]
    \begin{center}
      \includegraphics[width=12cm,height=17cm]{./images/WeatherUnderground}
      \caption{Consulta al servicio de WeatherUnderground}
    \end{center}
    \end{figure}
    \paragraph{Finalmente, si el modelo de datos aún no está completo, se consulta a la API de Forecast.io para buscar los datos faltantes. Esta es la última opción de búsqueda ya que la API tiene un número limitado de consultas por día.}


  \subsection{Dificultades}
    \paragraph{El desarrollo del módulo tuvo su complejidad en la obtención de los datos en las diferentes fuentes 
las fuentes de datos que tienen estructuras muy variadas, por lo que para tener un primer avance se han programado las pruebas unitarias y la funcionalidad para la extracción y estandarización de la información de cada fuente.}
  

  \subsection{Tecnologías de desarrollo}
  \paragraph{Para el desarrollo del módulo se hizo uso del lenguaje de programación Groovy.} 

  \paragraph{Smart Owl es un microservicio que se despliega con ayuda de SpringBoot y expone dos urls: /weather y /pollution que reciben los parámetros \textbf{latitude} y \textbf{longitude} para la búsqueda de la información.}

  \paragraph{Se escribió un conjunto de pruebas unitarias para la funcionalidad de la obtención de datos y la estandarización de la información con la ayuda de Spock Framework.}
  \paragraph{Gradle fue útli para las tareas de testing, administración de dependencias y la implementación del framework SpringBoot para el despliegue de la aplicación.}

  \newpage
  \begin{landscape}
      \subsubsection{Diagrama por bloques.}
        \paragraph{A continuación se mostrará el diagrama por bloques que define la estructura de Smart Owl.}
        \begin{figure}[b!]
        \centering
        \includegraphics[width=22.5cm,height=12cm]{./images/DiagramaAncientTortoise.png}
        \caption{Diagrama General de Smart Owl}
      \end{figure}
      \end{landscape}
      \newpage
    \paragraph{A continuación se muestran los diagramas de secuencia planteados para el funcionamiento del módulo mencionado.}
    \begin{center}
      \includegraphics[width=14cm,height=9cm]{./images/SmartOwlSequenceDiagram}
    \end{center}
    \begin{center}
      \includegraphics[width=14cm,height=9cm]{./images/SmartOwlSequenceDiagram2}
    \end{center}
    \subsubsection{Diagrama por bloques}
