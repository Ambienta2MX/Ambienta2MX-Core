\clearpage
\chapter{Modelo de Datos}

\section{Justificación}
Se hará uso de un modelo de base de datos orientado a documentos. Utlizarémos una base de datos no relacional MongoDB que guarda estructuras de datos en documentos JSON. Esta base de datos pertenece a la categoria NoSQL. \\
\newline
Las características de una base de datos NoSQL son las siguientes:

\begin{itemize}
  \item Modelo de datos flexible
  \item Buen rendimiento en clusters 
  \item Sin esquemas
\end{itemize}

Este tipo de bases nos resultará últi debido a que se procesarán grandes volúmenes de información para mostrar un historial de la información climática que se vaya persistiendo. 

Se escogió una base de datos orientada a documentos ya que la aplicación está orientada al análisis de datos y la obtención de información en tiempo real; además de que el uso de transacciones complejas no será necesario. 

\section{Descripción}
Para la persistencia de la información hemos definido los siguientes 3 documentos:   
\begin{itemize}
  \item Places
  \item Pollution
  \item Weather  
\end{itemize}

\subsection{Places}


\subsection{Pollution}


\subsection{Weather}
