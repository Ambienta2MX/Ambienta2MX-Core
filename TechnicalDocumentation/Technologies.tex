\clearpage
\chapter{Tecnologías}
\section{Ventajas y Desventajas}

\begin{center}
  \begin{tabular}{ | c | p{6cm} | p{6cm} | }
    \hline
    Nombre & Ventajas & Desventajas \\
    \hline
    MongoDB & \tabitem Implementa funciones espaciales para encontrar información relevante de ubicaciones específicas. & \tabitem Difícultad para dar seguimiento a los cambios ya que la estructura de la base cambia constantemente. \\
            & \tabitem & \tabitem  \\
    \hline
    Groovy & \tabitem Lenguaje expresivo. & \tabitem \\
           & \tabitem Se pueden integrar fácilmente todas las bibliotecas de Java ya que corre sobre la JVM. & \tabitem \\
           & \tabitem Closures & \\
    \hline
    Gradle  & \tabitem Sigue un enfoque de construcción por convención. & \tabitem La curva de aprendizaje es pronunciada si no se han utilizado antes herramientas como Ant o Maven \\
            & \tabitem Utiliza un poderoso lenguaje específico de dominio. (Groovy) & \\
            & \tabitem Cuenta con un administrador de dependencias que se encarga de 
                       descargarlas y las deja disponibles para su uso en la aplicación. & \\
    \hline

    Gulp  & \tabitem  Permite la automatización y ejecución de tareas para la construcción de un proyecto JavaScript.& \\
    \hline
    Yeoman & \tabitem Define la estructura de directorios y archivos para un proyecto JavaScript & \\
           & \tabitem Cuenta con varios generadores para crear el código repetitivo que se necesita para iniciar un proyecto. & \\
           & \tabitem Define tareas para que el desarrollador se concentre sólo en realizar la funcionalidad de la aplicación. & \\
    \hline
    Ember & \tabitem Incrementa la productividad al desarrollar aplicaciones single page. & \tabitem Curva de aprendizaje muy pronunciada. \\
          & \tabitem Ember Data. & \tabitem \\
          & \tabitem & \tabitem \\
    \hline
    PhantomJS & \tabitem & \tabitem \\
    \hline
  \end{tabular}
\end{center}
