\chapter{Tecnologías}
\paragraph{Para el desarrollo del proyecto decidimos utilizar un lenguaje de programación dinámico que corre sobre la JVM (Groovy) y el conjunto de herramientas que existen en su ecosistema. Además, para la parte de la aplicación web será necesario integrar tecnologías JavaScript que nos ayuden en la interacción con la aplicación que expone datos para mostrar la información que se requiera.}

\paragraph{A continuación se describen las ventajas de las tecnologías que hemos decidido utilizar para el desarrollo del proyecto}

\begin{table}[h]
\begin{center}
  \begin{tabular}{ | c | p{10cm} | }
    \toprule
    Nombre & Ventajas \\
    \midrule
    \raisebox{-\totalheight}{\includegraphics[width=0.3\textwidth, height=30mm]{images/mongo}} &
      \begin{itemize}[topsep=0pt]
        \item Implementa funciones espaciales para encontrar información relevante de ubicaciones específicas.
        \item Útil para el procesamiento de grandes cantidades de información. 
      \end{itemize} \\
    \midrule    

    \raisebox{-\totalheight}{\includegraphics[width=0.3\textwidth, height=25mm]{images/groovy}} &
     \begin{itemize}[topsep=0pt]
        \item Lenguaje expresivo que incrementa la productividad. Se reduce el azucar sintactico en comparación con lenguajes como C++ y Java y la curva de aprendizaje es muy pequeña si ya se conoce Java.
        \item Se pueden integrar fácilmente todas las bibliotecas de Java ya que corre sobre la JVM. 
        \item Closures 
      \end{itemize} \\
    \midrule
     
    \raisebox{-\totalheight}{\includegraphics[width=0.3\textwidth, height=15mm]{images/gradle}} &
    \begin{itemize}[topsep=0pt]
      \item Sigue un enfoque de construcción por convención.
      \item A diferencia de otras herramientas para la construcción de proyectos como Maven o Ant que utilizan XML, Gradle hace uso de un poderoso lenguaje específico de dominio (Groovy) para la definición de las tareas que deben ejecutarse.
      \item Cuenta con un administrador de dependencias que se encarga de descargarlas y las deja disponibles para su uso en la aplicación.
    \end{itemize} \\
    \midrule
      \raisebox{-\totalheight}{\includegraphics[width=0.3\textwidth,height=1.6cm]{images/grails}} &
        \begin{itemize}[topsep=0pt]
          \item Genera la estructura lógica para aplicar el patrón de arquitectura Model Vista Controlador, ya que se encuentra construido sobre Spring MVC.
          \item Buen soporte de pruebas unitarias, de integración y funcionales.
          \item Capa de Mapeo Objeto Relacional que trabaja sobre Hibernate para las operaciones transaccionales. 
        \end{itemize} \\
    \midrule
      \raisebox{-\totalheight}{\includegraphics[width=0.3\textwidth,height=1.5cm]{images/vertx}} &
      \begin{itemize}[topsep=0pt]
          \item Gran escalabilidad 
          \item Canal distribuido de mensajes como medio de comunicación.
          \item Herramienta políglota. 
      \end{itemize} \\
    \bottomrule
  \end{tabular}
\end{center}
\end{table}

\clearpage
\begin{table}
\begin{center}
  \begin{tabular}{ | c | p{12cm} | }
    \toprule 
      \raisebox{-\totalheight}{\includegraphics[width=1.5cm,height=3cm]{images/gulp}} &
        \begin{itemize}[topsep=0pt]
          \item Permite la automatización y ejecución de tareas para la construcción de un proyecto JavaScript.  
        \end{itemize} \\
    \midrule
      \raisebox{-\totalheight}{\includegraphics[width=0.3\textwidth, height=35mm]{images/yeoman}} &
      \begin{itemize}[topsep=0pt]
        \item Define la estructura de directorios y archivos para un proyecto JavaScript.
        \item Cuenta con varios generadores para crear el código repetitivo que se necesita para iniciar un proyecto.
        \item Define tareas para que el desarrollador se concentre sólo en realizar la funcionalidad de la aplicación.
      \end{itemize} \\
    \midrule
      \raisebox{-\totalheight}{\includegraphics[width=0.3\textwidth, height=35mm]{images/ember}} &
      \begin{itemize}[topsep=0pt]
        \item Incrementa la productividad al desarrollar aplicaciones single page.
        \item Permite la escalabilidad de una aplicación JavaScript.
        \item Un modelo de objetos rico y plantillas que se enlazan a los datos.
        \item Cuenta con propiedades calculadas de un modelo de datos. 
      \end{itemize} \\
    \midrule
      \raisebox{-\totalheight}{\includegraphics[width=0.3\textwidth, height=35mm]{images/phantomjs}} &
      \begin{itemize}[topsep=0pt]
        \item Automatización del navegador para realizar pruebas funcionales.
        \item Opción para hacer screenshots de algún flujo de la aplicación.
      \end{itemize} \\
   \bottomrule 
  \end{tabular}
  \caption{Descripción de las tecnologías}
  \label{Descripción de las tecnologías}
\end{center}
\end{table}
