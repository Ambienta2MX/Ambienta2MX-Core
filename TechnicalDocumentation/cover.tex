%%  COVER
\pagenumbering{Alph}
\begin{titlepage}
    \begin{center}
    \begin{tabular}{r c l}
    \includegraphics[scale=.20]{images/ipn} & \textbf{INSTITUTO POLIT\'ECNICO NACIONAL} & \includegraphics[scale=.20]{images/escom}\\ 
    & \textbf{ESCUELA SUPERIOR DE C\'OMPUTO}
    \end{tabular}
    \end{center}


    \vspace{1.5cm}
    \begin{center}
    \large Trabajo Terminal: \linebreak

    \large \textbf{``Ambienta2MX''} \linebreak
    \large 2014B-073

    \end{center}

    \vspace{1.5cm}

    \begin{center}
    Presentan: \linebreak
    \textbf{Jimenez García Eduardo Gamaliel} \linebreak
    \textbf{Reséndiz Arteaga Juan Alberto} \linebreak
    \end{center}

    \vspace{1.5cm}


    %En el presente documento se encuentran los resultados correspondientes al desarrollo del Trabajo Terminal cuyo objetivo es la implementaci\'on de un sistema que analice mensajes mediante  procesamiento de lenguaje natural y teoría de reconocimiento de patrones. Este sistema pretende servir como herramienta capaz de clasificar la intención de dichas  conversaciones como peligrosas o no peligrosas. \linebreak

    \textbf{Palabras Clave}:  Desarrollo Web, Sistemas distribuidos, Aplicaciones para las comunicaciones en red.

    \vspace{1.5cm}
     
    \begin{center}


    Directores: \linebreak
    \textbf{ M. en C. Ram\'irez Morales Mario Augusto, M. en E.  Silva S\'anchez Carlos}

    \end{center}
\end{titlepage}
\pagenumbering{arabic}