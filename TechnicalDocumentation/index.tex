\tableofcontents
\clearpage
\section{Tecnolog�as}
\paragraph{MogoDB} ~\\
\newline
Se har� uso de una base de datos no relacional para definir el modelo de datos y persistir la informaci�n que se obtenga de los
servicios expuestos por las instituciones.

\paragraph{Groovy} ~\\
\newline
El lenguaje de programaci�n din�mico Groovy se emplear� para el tratamiento de la informaci�n expuesta por las instituciones y
la creaci�n de un nuevo est�ndar para esos metadatos.

\paragraph{Gradle} ~\\
\newline
Grade es una herramienta de construcci�n de proyectos que se utilizar� para la administraci�n de las dependencias y la 
automatizaci�n de tareas, como el despliegue de los m�dulos o la ejecuci�n de procesos para la obtenci�n y estandarizaci�n de los metadatos.

\paragraph{Gulp} ~\\
\newline
Gulp nos ayudar� en la automatizaci�n de las tareas para la construcci�n y el despliegue de la aplicaci�n en la que cualquier usuario
podr� consultar los metadatos a trav�s de los servicios que expongamos.

\paragraph{Yeoman} ~\\
\newline
Yeoman es una herramienta que define una estructura para aplicaciones JavaScript. Ser� �til para generar la estructura b�sica de una 
aplicaci�n que trabaje con Gulp.

\paragraph{Ember} ~\\
\newline
Es un framework para el desarrollo de single page applications. Lo utilizaremos para la estructura de la aplicaci�n que va del 
lado del cliente ya que sus convenciones har�n m�s escalable el desarrollo.

\paragraph{PhantomJS} ~\\
\newline
Cuando un flujo de la aplicaci�n este completo se implementar�n un conjunto de pruebas funcionales haciendo uso de PhantomJS 
para verificar si el flujo es correcto.
