\newpage
\section*{Glosario}
\begin{itemize}
	\item \textbf{JSON}: Javascript Object Notation, Es un formato ligero de intercambio de datos.
	\item \textbf{ITRF}: Acrónimo de International Terrestrial Reference System (Marcos de Referencia Terrestre Internacional).
	\item \textbf{NAD27}: Acrónimo de North American Datum of 1927. Marco de referencia terrestre usado por el INEGI hasta el año 1998.
	\item \textbf{GIS}: Sistema de Información Geográfica.
  \item \textbf{API}: Application Programming Interface, conjunto de rutinas, protocolos o herramientas para construcción de software.
  \item \textbf{REST}: Representational State Transfer, Arquitectura de sofware para sistemas web basada en un patrón ``Convention over configuration'' con operaciones que funciona usando el protoclo HTTP como base.
  \item \textbf{SMN}: Servicio Meteorológico Nacional.
  \item \textbf{CONAGUA}: Comisión Nacional del Agua.
  \item \textbf{INEGI}: Instituto Nacinal de Estadística y Geografía.
  \item \textbf{Políglota}: En área informática, que tiene soporte o comprende varios lenguajes de programación.
  \item \textbf{StakeHolder}: Personas u organizaciones que participan o tienen un interes en un proceso específico o definido dentro de un empresa. Se convierten en fuente clave de información para desarrollar un sistema o mejorar algún proceso.
  \item \textbf{SaaS}: Acrónimo de Software as a Service. Es un modelo de distribución de software donde el soporte lógico y los datos que maneja se alojan en servidores de una compañía de tecnologías de información y comunicación (TIC).
\end{itemize}
\addcontentsline{toc}{chapter}{Glosario}